%        File: cv.tex
%     Created: mar. nov. 09 08:00  2010 C
% Last Change: mar. nov. 09 08:00  2010 C
%

\documentclass[11pt,a4paper,sans]{moderncv}

% moderncv themes
\moderncvtheme[blue]{classic}                 % optional argument are 'blue' (default), 'orange', 'red', 'green', 'grey' and 'roman' (for roman fonts, instead of sans serif fonts)
%\moderncvtheme[green]{classic}                % idem

% character encoding
\usepackage[utf8]{inputenc}                   % replace by the encoding you are using
\usepackage[T1]{fontenc}

% adjust the page margins
\usepackage[scale=0.8]{geometry}
%\setlength{\hintscolumnwidth}{3cm}						% if you want to change the width of the column with the dates
%\AtBeginDocument{\setlength{\maketitlenamewidth}{6cm}}  % only for the classic theme, if you want to change the width of your name placeholder (to leave more space for your address details
%\AtBeginDocument{\recomputelengths}                     % required when changes are made to page layout lengths
\addtolength{\textheight}{160pt} 						 % triche pour rester sur une page

\newcommand{\superscript}[1]{\ensuremath{^{\textrm{#1}}}}

\firstname{Laurent}
\familyname{Hory}
\title{Ingénieur d’étude et développement}
\address{2 rue d'Espagne}{91300 Massy}
\mobile{06 71 80 11 41}
\email{laurent.hory@gmail.com}
\extrainfo{Nationalité Française \\ Né le 24 juin 1985 -- 30 ans} % optional, remove the line if not wanted

\makeatletter
\renewcommand*{\bibliographyitemlabel}{\@biblabel{\arabic{enumiv}}}
\makeatother


\begin{document}
\maketitle

\section{Formation}
\cventry{2009 -- 2011}{Master Informatique}{Université Henri Poincaré}{Nancy}{}{Spécialité Reconnaissance Apprentissage et Raisonnement.}
\cventry{2005 -- 2009}{Licence Mathématiques \& Informatique}{Université Henri Poincaré}{Nancy}{}{Spécialité Informatique.}% arguments 3 to 6 can be left empty
\cventry{2003 -- 2005}{BTS IRIS}{Lycée Raoul Follereau}{Belfort}{}{Informatique et Réseaux pour l'Industrie et les Services.}
\cventry{2001 -- 2003}{Baccalauréat STI}{Lycée Lumières}{Luxeuil-les-Bains}{}{Génie électronique.}

\section{Expérience}
\cventry{Février 2016 \\ Aujourd'hui}{DevOps}{General Electric consultant Altran}{Buc}{}{
Migration de l'intégration continue de Coverity 4 vers Coverity 7.
Création d'environnements de développement sur machine virtuelle.
Automatisation de la génération des machine de build avec Ansible.
Conteneurisation de chaine de compilation pour une architecture en micro services reposant sur Docker et Perforce.
}
\cventry{Septembre 2014 \\ Octobre 2015}{DevOps}{General Electric consultant ATos}{Buc}{}{
Support poste Linux et machine virtuelles (CentOS, VMWares/VSphere)
Maintenance d'outil de déploiement (javascript, JQuery, PHP)
Réécriture de l'interface de l'outil de déploiement (MongoDB, Express, AngularJS, Nodejs)
Mise en place d'intégration continue à l'aide de Jenkins sur deux projets.
Migration d'un projet de Clearcase vers Mercurial.
}
\cventry{Octobre 2011 \\ (4 ans)}{Ingénieur d'études}{General Electric consultant Atos}{Buc}{}{
Amélioration d'un système de gestion de texte sur un ensemble d'images médicales
Amélioration d'interface graphique java
Refactoring de code C en C++
Optimisation des performance sur du code C/C++ et sur des scripts bash, tcsh, sh
Maintenance et débogage du code existant (gdb, valgrind)
Création de script d'automatisation des taches
Développement d'un framework de tests en Python
Écriture de tests automatisés
Animation de meeting d'intégration et réalisation de la release
}
\cventry{Mars 2011 \\ (6 mois)}{Ingénieur}{Dassault Systèmes}{Vélizy-Villacoublay}{Projet de fin d'études}{Ajout de fonctionnalités d'interactions et de comportement a un moteur physique. Utilisation de C++, lua, OpenCL, OpenGL, GLSL.}
\cventry{Février 2010 \\ (7 mois)}{Développeur}{Loria}{Nancy}{Projet d'initiation à la recherche de 4 mois, avec prolongation d'un stage d'été de 3 mois}{Réalisation d'un logiciel de visualisation de structure de billons d'épicéa. Conception logiciel à l'aide des Design Pattern et UML. Mise en place et optimisation des algorithmes avec C++, Qt et OpenGL.}
\cventry{Février 2009 \\ (5 mois)}{Développeur}{LCM3B}{Nancy}{Projet de validation de licence}{Réalisation d'un logiciel de visualisation moléculaire avec calcul des mailles cristallographiques. Conception de l'architecture à l'aide du langage UML, réalisation du programme avec le langage C++, le framework Qt et l'API OpenGL.}
\cventry{Janvier 2005 \\ (6 mois)}{Développeur}{}{Belfort}{Projet de validation de BTS}{Réalisation d'un espace web et d'un logiciel de gestion de pas de tir pour un club de tir à l'arc. Utilisation des technologie PHP, Javascript, C++, MySQL.}
\newpage
\section{Compétences}
\cvcomputer{\textbf{D}éveloppement:}{C, C++, Python, PHP, Java, JavaScript, SQL, Qt, XHTML/CSS, OpenGL, OpenCL, AngularJS, JQuery}{\textbf{B}ase de données:}{Oracle, Mysql, PostgreSQL, MongoDB}
\cvcomputer{\textbf{G}estion de version:}{Mercurial, Git, Subversion, Perforce, clearcase}{\textbf{S}ysteme:}{Docker, VSphere, Jenkins, Ansible, GlusterFS}
\cvcomputer{\textbf{S}ystèmes d'exploitation:}{GNU/Linux, Windows}{\textbf{L}ogiciel, IDE, Outils:}{Vim, Eclipse, Netbeans, Qt Creator, Emacs, Visual Studio, OpenOffice, \LaTeX, Scilab, Gdb, Valgrind, Kcachegrind, IBM Rational ClearQuest, Jira}

\section{Intérêts}
\cvline{Loisirs:}{Jeux de société, musique rock.}

\end{document}
